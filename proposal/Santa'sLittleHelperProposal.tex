\documentclass[a4paper,12pt]{article}


\usepackage[english]{fancyref}
\usepackage{fullpage}
%\usepackage[top=1in, bottom = 1in, left = 1in, rigth = 1in]{geometry}
\usepackage[english]{babel}
\usepackage[utf8x]{inputenc}
\usepackage{amsmath}
\usepackage{graphicx}
\usepackage{float}
\usepackage[colorinlistoftodos]{todonotes}
\bibliographystyle{ieeetr}
\newcommand{\RNum}[1]{\uppercase\expandafter{\romannumeral #1\relax}}
\usepackage{tocbibind}
\usepackage{setspace}


\author{Zhenning Jiang (z5082223)}
\date{\today}
%-------Macros--------------------
\newcommand{\CCM}{continuous conduction mode}


\begin{document}
	\begin{titlepage}
		\begin{center}
			\Huge{
				ELEC9782 Mobile Applications and Network Performance\\
			}
			\Huge{
				Project Proposal\\
				[1cm]
				Application: Santa's Little Helper\\
			}
			\\[2cm]\\
			
			\includegraphics[width = 0.4\textwidth]{UNSW_coat_of_arms.png}\\
			[1.5cm]
			\Large{Group Member: \\
				Chayut Orapinpatipat\\
				Ken Cheung \\
				Jingming Yang\\
				Jinhui Li\\
				Bowen Gu\\
				Zhenning Jiang (z5082223)\\
				[0.7cm]
				\today\\
				[0.7cm]
			}		
		\end{center}
	\end{titlepage}
    \newpage
	
	\section{Summary}
	Santa's little helper is aiming at becoming the lift assistant of the users. It provides location or event based notification services. For example, user can predefine an outgoing email to send to persons who is going to have meeting with the second day. In case of the user miss the wake up alarm, the application can automatically detect that the user is inactive and send out the predefine email to inform the persons of possible delay or cancel of the meeting.
	
	
	
	\section{Features}
	
	
\end{document}